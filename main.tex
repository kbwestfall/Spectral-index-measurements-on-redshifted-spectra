\documentclass[a4paper]{article}

%% Language and font encodings
\usepackage[english]{babel}
\usepackage[utf8x]{inputenc}
\usepackage[T1]{fontenc}

%% Sets page size and margins
\usepackage[a4paper,top=3cm,bottom=2cm,left=3cm,right=3cm,marginparwidth=1.75cm]{geometry}

%% Useful packages
\usepackage{amsmath}
\usepackage{graphicx}
\usepackage[colorinlistoftodos]{todonotes}
\usepackage[colorlinks=true, allcolors=blue]{hyperref}

\title{Spectral index measurements on redshifted spectra}
\author{Kyle Westfall}

\begin{document}
\maketitle

\section{Bandhead indices integrating over $F_\nu$}

D4000 is defined as:
%
\begin{equation}
{\rm D4000} = \frac{\langle F_{\nu,{\rm red}}\rangle}{\langle F_{\nu,{\rm blue}}\rangle}
\end{equation}
%
where
%
\begin{equation}
\langle F_{\nu} \rangle = 
\left(\int^{\lambda_{2}}_{\lambda_{1}} F_\nu\ d\lambda\right) \cdot \left(\int^{\lambda_{2}}_{\lambda_{1}} d\lambda\right)^{-1}.
\end{equation}
%
The units of $F_{\nu}$ are erg/s/cm$^2$/Hz, meaning that the redshifted flux density are different from the observed flux density, even when accounting for the difference in the wavelength of the observations.  That is,
%
\begin{eqnarray}
F_{\nu,{\rm obs}} & = & F_{\nu,{\rm rest}}\ \frac{\nu_{\rm rest}}{\nu_{\rm obs}} \\
 & = & F_{\nu,{\rm rest}}\ (1+z)
 \end{eqnarray}
%
If I then also use
%
\begin{eqnarray}
\lambda_{\rm obs} & = & (1+z)\ \lambda_{\rm rest} \\
d\lambda_{\rm obs} & = & (1+z)\ d\lambda_{\rm rest},
\end{eqnarray}
%
I can rewrite the mean flux per Hz in both the observed and rest frame.  In the observed frame:
%
\begin{eqnarray}
\langle F_{\nu,{\rm obs}} \rangle & = &  
\left(\int^{\lambda_{2,{\rm obs}}}_{\lambda_{1,{\rm obs}}} F_{\nu,{\rm obs}} d\lambda_{\rm obs}\right) \cdot \left( \int^{\lambda_{2,{\rm obs}}}_{\lambda_{1,{\rm obs}}} d\lambda_{\rm obs} \right)^{-1} \\
& = &  
\left( (1+z)^2\ \int^{\lambda_{2,{\rm rest}}}_{\lambda_{1,{\rm rest}}} F_{\nu,{\rm rest}}\ d\lambda_{\rm rest}\right) \cdot \left( (1+z)\ \int^{\lambda_{2,{\rm rest}}}_{\lambda_{1,{\rm rest}}} d\lambda_{\rm rest}\right)^{-1} \\
& = & (1+z) \langle F_{\nu,{\rm rest}} \rangle,
\end{eqnarray}
%
which is really just a repeat of equation 4.  If all of that makes sense, then it should be that:
%
\begin{eqnarray}
{\rm D4000}_{\rm obs} & = & \frac{\langle F_{\nu,{\rm red,obs}}\rangle}{\langle F_{\nu,{\rm blue,obs}}\rangle} \\
 & = &  \frac{(1+z)\ \langle F_{\nu,{\rm red,rest}}\rangle}{(1+z)\ \langle F_{\nu,{\rm blue,rest}}\rangle} \\
 & = & {\rm D4000}_{\rm rest},
\end{eqnarray}
%
as long as I perform the $D4000_{\rm obs}$ measurements with the passbands appropriately redshifted.

\section{Bandhead indices integrating over $F_\lambda$}

The Conroy \& van Dokkum definition of the TiO index is:
%
\begin{equation}
{\rm TiO} = \frac{\langle F_{\lambda,{\rm blue}}\rangle}{\langle F_{\lambda,{\rm red}}\rangle},
\end{equation}
%
where the mean $F_\lambda$ is the same as equation 2 just replacing $F_\nu$ with $F_\lambda$.  Similar to equation 4, I can write:
%
\begin{equation}
F_{\lambda,{\rm obs}} = F_{\lambda,{\rm rest}} / (1+z),
\end{equation}
%
such that
%
\begin{eqnarray}
{\rm TiO}_{\rm obs} & = & \frac{\langle F_{\lambda,{\rm blue,obs}}\rangle}{\langle F_{\lambda,{\rm red,obs}}\rangle} \\
 & = &  \frac{\langle F_{\lambda,{\rm blue,rest}}\rangle/(1+z)}{\langle F_{\lambda,{\rm red,rest}}\rangle/(1+z)} \\
 & = & {\rm TiO}_{\rm rest},
\end{eqnarray}
%
as long as I perform the measurements within the redshifted passbands.

\section{Absorption-line indices in magnitudes integrating over $F_\lambda$}

Absorption-line indices that integrate over $F_\lambda$ and are in magnitude units, like CN1, are defined as:
%
\begin{equation}
{\rm CN1} = -2.5 \log\left[ \left( \int^{\lambda_2}_{\lambda_1} \frac{F_\lambda}{C_\lambda} d\lambda \right) \cdot \left(\int^{\lambda_2}_{\lambda_1} d\lambda\right)^{-1} \right],
\end{equation}
%
where $C_\lambda$ is the linear fit to the pseudocontinua on either side of the main band.  The pseudocontinua are measurements of $\langle F_\lambda\rangle$ in a blue and red band such that $C_\lambda = m\lambda + b$ where,
%
\begin{eqnarray}
m & = & (\langle F_{\lambda}\rangle_{\rm red} - \langle F_{\lambda}\rangle_{\rm blue}) (\langle \lambda\rangle_{\rm red} - \langle \lambda\rangle_{\rm blue})^{-1} \\
b & = & \langle F_{\lambda}\rangle_{\rm blue} - m\ \lambda.
\end{eqnarray}
%
Working through the math then $m_{\rm obs} = m_{\rm rest} (1+z)^{-2}$ and $b_{\rm obs} = b_{\rm rest} (1+z)^{-1}$, such that $C_{\lambda, {\rm obs}} = C_{\lambda, {\rm rest}}  (1+z)^{-1}$, as you would expect.  This means that $F_{\lambda,{\rm obs}}/C_{\lambda,{\rm obs}} = F_{\lambda,{\rm rest}}/C_{\lambda,{\rm rest}}$ such that,
%
\begin{eqnarray}
{\rm CN1}_{\rm obs} & = &  
-2.5\log\left[ \left(\int^{\lambda_{2,{\rm obs}}}_{\lambda_{1,{\rm obs}}} \frac{F_{\lambda,{\rm obs}}}{C_{\lambda,{\rm obs}}}\ d\lambda_{\rm obs}\right) \cdot \left( \int^{\lambda_{2,{\rm obs}}}_{\lambda_{1,{\rm obs}}} d\lambda_{\rm obs} \right)^{-1} \right] \\
& = & -2.5\log\left[ \left((1+z) \int^{\lambda_{2,{\rm rest}}}_{\lambda_{1,{\rm rest}}} \frac{F_{\lambda,{\rm rest}}}{C_{\lambda,{\rm rest}}}\ d\lambda_{\rm rest}\right) \cdot \left( (1+z) \int^{\lambda_{2,{\rm rest}}}_{\lambda_{1,{\rm rest}}} d\lambda_{\rm rest} \right)^{-1} \right] \\
& = & {\rm CN1}_{\rm rest}.
\end{eqnarray}

\section{Absorption-line indices in \AA\ integrating over $F_\lambda$}

Absorption-line indices, integrating over $F_\lambda$ and in units of \AA, like NaD, are defined as
%
\begin{equation}
{\rm NaD} = \int^{\lambda_2}_{\lambda_1} \left(1 - \frac{F_\lambda}{C_\lambda}\right) d\lambda .
\end{equation}
%
It follows from above then that
\begin{eqnarray}
{\rm NaD}_{\rm obs} & = & \int^{\lambda_{2,{\rm obs}}}_{\lambda_{1,{\rm obs}}} \left(1 - \frac{F_{\lambda,{\rm obs}}}{C_{\lambda,{\rm obs}}}\right) d\lambda_{\rm obs} \\
& = & (1+z) \int^{\lambda_{2,{\rm rest}}}_{\lambda_{1,{\rm rest}}} \left(1 - \frac{F_{\lambda,{\rm rest}}}{C_{\lambda,{\rm rest}}}\right) d\lambda_{\rm rest} \\
& = & (1+z)\ {\rm NaD}_{\rm rest} .
\end{eqnarray}
%

\section{Emission-line equivalent widths in \AA\ integrating over $F_\lambda$}

Emission-line equivalent widths, integrating over $F_\lambda$ and in units of \AA, are defined similarly to absorption-line indices, just with a sign change:
%
\begin{equation}
{\rm EW}_{{\rm H}\alpha} = \int^{\lambda_2}_{\lambda_1} \left(\frac{F_\lambda}{C_\lambda} - 1\right) d\lambda.
\end{equation}
%
However, in detail, the DAP actually measures
%
\begin{equation}
{\rm EW}_{{\rm H}\alpha} = \frac{F_{{\rm H}\alpha}}{C_{\lambda,c}},
\end{equation}
%
where
%
\begin{eqnarray}
F_{{\rm H}\alpha,{\rm obs}} & = & \int^{\lambda_{2,{\rm obs}}}_{\lambda_{1,{\rm obs}}} (F_{\lambda,{\rm obs}} - C_{\lambda,{\rm obs}})\ d\lambda_{\rm obs} \\
 & = & \int^{\lambda_{2,{\rm rest}}}_{\lambda_{1,{\rm rest}}} (1+z)^{-1}\ (F_{\lambda,{\rm rest}} - C_{\lambda,{\rm rest}})\ (1+z) d\lambda_{\rm rest} \\
 & = &  F_{{\rm H}\alpha,{\rm rest}}
\end{eqnarray}
%
and $C_{\lambda,c,obs} = C_{\lambda,c,rest}/(1+z)$ is the linear continuum sampled at the center of the emission line.  Therefore,
%
\begin{eqnarray}
{\rm EW}_{{\rm H}\alpha,{\rm obs}} & = & \frac{F_{{\rm H}\alpha,{\rm obs}}}{C_{\lambda,c,{\rm obs}}} \\
& = & (1+z)\ \frac{F_{{\rm H}\alpha,{\rm rest}}}{C_{\lambda,c,{\rm rest}}} \\
 & = & (1+z)\ {\rm EW}_{{\rm H}\alpha,{\rm rest}}.
\end{eqnarray}

\end{document}